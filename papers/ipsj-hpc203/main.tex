%%
%% 研究報告用スイッチ
%% [techrep]
%%
%% 欧文表記無しのスイッチ(etitle,eabstractは任意)
%% [noauthor]
%%

%\documentclass[submit,techrep]{ipsj}
\documentclass[submit,techrep,noauthor]{ipsj}

\usepackage[dvips]{graphicx}
\usepackage{latexsym}

\def\|{\verb|}

\begin{document}

\title{LocustaRPC: 次世代リーダーシップマシンのための\\スケーラブルなRPC基盤}

\etitle{LocustaRPC: A Scalable RPC Framework\\for Next-Generation Leadership Machines}

\affiliate{UT}{筑波大学\\
University of Tsukuba}

\author{中野 将生}{Masaki Nakano}{UT}
\author{前田 宗則}{Munenori Maeda}{UT}
\author{建部 修見}{Osamu Tatebe}{UT}

\begin{abstract}
次世代リーダーシップマシンでは数十万ノード規模のシステムが想定されており,
従来のRPC基盤ではスケーラビリティが課題となる.
本稿では,大規模システム向けのスケーラブルなRPC基盤であるLocustaRPCを提案する.
LocustaRPCは,RDMAを活用した低遅延通信と,
スケーラブルなコネクション管理機構を備える.
評価の結果,既存手法と比較して高いスケーラビリティを達成することを確認した.
\end{abstract}

\maketitle

\section{はじめに}

\section{関連研究}

\section{設計}

\section{実装}

\section{評価}

\section{おわりに}

\begin{acknowledgment}
謝辞をここに記述する.
\end{acknowledgment}

\bibliographystyle{ipsjsort}
\bibliography{refs}

\end{document}
